Several contributions are made in this study, to reveal an effective algorithm for glaucoma screening. Throughout the proposed method, a particular attention was assigned to the computational aspect, to provide an easy-to-perform and computationally-efficient method. Finally, we have obtained a final 98\% accuracy rate on glaucoma screening.

First, the prior OD detection tends to propose an efficient way to detect the OD area, among the eventual presence of bright lesions in pathological cases. These bright lesions, linked to ocular diseases such as diabetic retinopathy, are not related to the glaucoma case. However, it is mandatory to efficiently conduct glaucoma screening, even with a subject suffering from an ocular disease inducing the apparition of these bright lesions. A preprocessing step, followed by the combination of the detection of bright regions and a template matching allows to effectively detect the OD, in a computationally-efficient way. This stage of the method offers a robust location of the area of interest, without extracting the retinal vessels inducing a more important complexity.
Since the prior OD detection can be quite sensitive in pathological cases, and need to be effective to lead to glaucoma screening, it have been tested on ten relevant databases in this field, showing an excellent performance on both healthy and non-healthy images.
One limitation here is the decreased ability to detect the OD in critical cases, where the OD appears in darker images, with uneven illumination or weak contrast. However, we consider that the OD seems to be brighter than the retinal background, and the method was designed with this assumption.

Second, a new method for joint OC-OD segmentation have been proposed. The main challenge was to perform an accurate and inexpensive segmentation of the areas. Here, an unsupervised classification method is firstly employed, using the texture-based information and assigning each pixel within a retinal component. From there, to improve the segmentation accuracy, a boundary fitting algorithm based on the circular Hough transform is performed. The main advantage of the proposed segmentation method is its accuracy on finding each pixel belonging to the desired areas. However, the method relies on the intensity feature, which can be not obvious in extreme cases. Nevertheless, applying our method from 2-D retinal images by combining intensity information and the boundary fitting allows to converge toward a precise approximation of the areas.

Third, the clinical CDR feature is used to assess the ONH structural changes, and detect the eventual presence of the glaucoma disease.
The CDR have been extensively used for this purpose, because of its clinical reliability. Here, we computed the area-based CDR, providing a 2-D evaluation of the cupping within the ONH. Then, we applied a thresholding on the CDR value to classify each retinal image belonging to the healthy class or the glaucomatous class. Following clinical studies and existing methods for this purpose, the thresholding value $T$ have been chosen regarding to the within-class variance. In our specific study, a fixed value $T$ is defined, to finally assign each retinal image among a class and give a diagnosis. Nevertheless, it can be useful to exploit the obtained mean and standard deviation values within each class, and exploit it in a different manner, such as proposing a more-effective way to assess glaucoma early or even grading the spreading of the glaucoma disease.
In our study, a specific value $T$ have been fixed, and we focused on comparing each assignment to the ground-truth result, in order to evaluate the effectiveness on final glaucoma screening. Anyway, by performing the prior OD detection and OC-OD segmentation, the clinical CDR feature allows to detect and identify any potential subject suffering from an early occurring glaucoma.

In this work, a fully-automated method was proposed, using traditional approaches with well-known computer vision algorithms. The new emerging deep learning approaches provide an end-to-end classification, automatically extracting features from the image  through a prior learning with a certain amount of labelled data. Nowadays, these new algorithms are extensively used in medical image analysis, bringing an invaluable help for the screening of ocular diseases for instance \citep{gulshan}. These cutting-edge algorithms have been exploited by recent state-of-the-art methods for glaucoma screening, unveiling interesting results in the assessment of the disease \citep{norouzifard, zilly}.