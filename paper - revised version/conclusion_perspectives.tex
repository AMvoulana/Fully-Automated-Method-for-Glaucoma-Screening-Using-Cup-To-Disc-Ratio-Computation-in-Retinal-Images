In this paper, a new approach for glaucoma screening and diagnosis was proposed. Using the CDR clinical feature, the method automatically detects the presence of glaucoma, from 2-D retinal images. We focused on the computational aspect throughout the proposition of the algorithm. A four-stages strategy was followed: robust OD detection in healthy and pathological cases, joint OC-OD segmentation framework using texture-based and model-based criteria, area-based CDR calculation, and final glaucoma classification between healthy and non-healthy subjects.
Using the presented evaluation databases and performance metrics, the proposed glaucoma screening approach reached a 98\% accuracy rate on final screening, outperforming the state-of-the-art methods for this purpose. 
Since the approach offers excellent performance rates, it can be employed on a large-scale screening program of the glaucoma disease, and be incorporated in computer-aided diagnosis systems for ophthalmologists. \\

At the present time, we are working on the implementation of the proposed method for glaucoma screening on smartphone-captured fundus images. Starting from the retinal image acquisition with a dedicated device, the aim is to assess glaucoma on portable platforms as an app, from the OD detection to the final diagnosis between healthy or potentially suffering from glaucoma. Hence, the system can be employed by medical specialists, to ensure their final screening of the disease.

In future work, we propose to explore deep learning algorithms to deliver improved, cutting-edge systems for early glaucoma screening and diagnosis. We aim to provide a precise assessment of the retinal strutures, to help ophthalmologists and clinicians reinforce their diagnosis and monitor the potential presence of the disease during time.



%In future work, we tend to overcome the encountered difficulties and propose to exploit deep learning methods to develop a fully-automatic system for glaucoma screening and diagnosis. Convolutional Neural Networks (CNN) have overcome several limitations of traditional computer vision algorithms, and outperformed many results on help-diagnosis systems. Thus, deep learning algorithms allows to avoid the critical stages of OD detection or OC and OD segmentation, and the use of features for glaucoma assessment. Hence, we will focus on how the architecture learn the glaucomatous behaviour in order to handle extreme cases then improve acquired performance rates. Larger datasets could also validate a more effective glaucoma screening system. Also, a perspective consists in the classification of the different glaucoma stages, from early, moderate and severe stages. 
