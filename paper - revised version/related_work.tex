Three main clinical techniques lead to glaucoma screening: assessment of the intraocular pressure with a tonometer, assessment of the visual field with static and kinetic tests, and assessment of the optic nerve head (ONH) with recent devices such as optical coherence tomography (OCT) or scanning laser ophthalmoscope (SLO) \citep{lim}. 
Nevertheless, these techniques suffer from disadvantages such as operator-dependence or significant  costs \citep{liu}.
To address these limitations, recent studies have proposed automated methods for glaucoma screening and diagnosis from retinal images \citep{bock2010,singh2}. These approaches tend to be more accurate in final disease screening, while being less expensive and contributing to the development of widespread glaucoma screening programs \citep{abramoff}.

Since the OC excavation is the first visible sign of the presence of glaucoma, the retinal image study for glaucoma assessment consists in the evaluation of the ONH morphological changes. 
Hence, different parameters such as diameter and area of the OC and the OD, or the area of the Neuro-Retinal Rim (NRR) need to be retrieved \citep{hu}. Then, clinical features such as the Inferior-Superior to Nasal-Temporal (ISNT) rule \citep{isnt} or the cup-to-disc ratio (CDR) \citep{cdr} are employed to assess glaucoma.
Among these different features, the CDR is a reliable and often-used clinical feature for early glaucoma screening and  diagnosis \citep{ophthal1}. It represents the ratio between the OC and the OD, according to one of the extracted parameters (diameter or area). The CDR value increases as the disease develops, and becomes higher than around 0.6-0.7 when the patient suffers from a greater risk of developing glaucoma \citep{abramoff}. Thereby, the CDR value helps to monitor the progression of glaucoma over time, to finally screen the disease early \citep{cdr}.

To compute the CDR from retinal images, a joint OC-OD segmentation is required \citep{cheng}. Accurate segmentation is mandatory to correctly calculate the CDR and finally classify healthy and glaucomatous subjects \cite{singh2}.
In the context of the project we are carrying out, the existing segmentation methods for glaucoma screening are generally ranged into two different approaches: supervised approaches and unsupervised approaches.

In supervised approaches, image-level features are extracted from the fundus images \citep{wong}. Then, a set of retinal images with the manual boundaries of the OC and the OD is used to train a specific classifier then detect the OC and OD areas with the computed features.
For example, \citet{chakravarty} proposed a boundary-based method for OC-OD segmentation, where several image features such as color gradients and depth estimations are retrieved from retinal images. A Conditional Random Field is formulated, using an energy minimization criteria to fit on the boundaries. Then, these features are used to feed a support vector machine (SVM) classifier during the learning phase. After obtaining the OC and OD boundaries, CDR calculation is finally performed to lead to glaucoma screening.
In the same way, \citet{cheng} extracted both OC and OD areas using a superpixel classification-based method. A simple linear iterative clustering (SLIC) superpixel  algorithm \citep{slic} is operated on the OD sub-image. Then, five color channel maps and their associated histograms are generated, as well as 18 center surrounded statistics maps to bring a more textural information. The features of each superpixel are used as the inputs of an SVM algorithm, to classify each superpixel as part of the OC or the OD. After finding each area, the CDR is computed to assess glaucoma.
%\citet{zilly} used an ensemble learning approach to segment the OC and OD areas. Entropy sampling is performed to extract the most informative pixels from the image. Then, convolutional neural networks are implemented for feature learning. Final segmentation is obtained by considering the convex hull of each detected area, and CDR calculation leads to glaucoma screening.
As a major advantage, the supervised approaches tend to efficiently segment the OC and OD areas. However, a supervised learning phase is required to perform the segmentation, inducing time-consuming or large data requirements when the precise ground-truth labelling is a tedious task. Moreover, the choice of the image features (color, texture, energy, etc.) used to train the segmentation classifier is difficult. These aspects of the supervised approaches may be an obstacle according to our purpose, where a low-computational algorithm is required. 

Unsupervised approaches provide OC and OD segmentation without any learning phase. Among unsupervised state-of-the-art approaches, image processing techniques such as image thresholding or morphological operators have been frequently used to segment both OC and OD areas \citep{aquino,stapor}. Then, CDR calculation leads to glaucoma screening, as a binary classification between healthy and glaucomatous patients is generally operated \citep{singh2}.
For example, \citet{singh} associated image pre-processing techniques, such as binarization and morphological operations to finally compute the CDR with the found OC and OD areas. Thereafter, a compensated CDR value allows to classify the retinal image as healthy or suspicious. Also, \citet{guerre} proposed a fully-automated method for glaucoma detection, based on basic processing techniques. A preprocessing phase is operated to manage with uneven illumination or eventual noise. After, a blood vessels' mask generated with an Isotropic Undecimated Wavelet Transform (IUWT) is combined with an adaptive thresholding algorithm to segment the OC and OD areas. Then, the CDR is computed as a feature to assess glaucoma.
In these methods, simple and effective algorithms are used to extract the desired areas, going with less computationally complex algorithms for glaucoma screening. However, since the segmentation task is arduous due to blood vessels occlusions, or variable imaging acquisitions such as weak illumination or low contrast, these algorithms may not be sufficient and provide underestimated extracted areas. Hence, a lack on segmentation performance is noticed, what influences the obtained CDR value and final glaucoma screening. 
To improve the segmentation accuracy using these low-complex unsupervised algorithms, a strategy consists in the use of model-based algorithms to fit on the boundaries.
In this way, some methods have used active contours \citep{joshi}. For example, \citet{cdr} formulated an approach mainly based on a variational level-set technique to detect the OC and OD borders. A thresholding method is applied to initialize the contour. Then, the CDR is computed from the detected areas to assess glaucoma.
The active contour based methods slightly improve the segmentation accuracy, compared to the methods based on image processing techniques only. However, the choice of the active contours parameters is challenging, such as the initialization around the OD or the contour deformation (with energy-based features for example), to effectively converge and fit on the desired boundaries. 
To avoid the use of deformable contours, some methods have used the circular Hough transform \citep{pedersen} to model the OC and OD boundaries. For example, \citet{priyadharsini} performed image processing techniques such as channel extraction or histogram equalization, to emphasize the intended areas in the retina. Then, circular Hough transform is operated to detect the OD boundaries. Next, morphological operations are subsequently used to detect the OC area. As a major advantage compared to active contours, circular Hough transform requires a few parameters to effectively find the OC and the OD, while providing a good computationally-efficient property against accuracy on the final segmentation result. In this work, we follow this direction, 
hence, we combine the Hough transform with a prior detection of the OD surroundings to answer to the drawbacks of the existing methods and detect the borders in a precise manner for further glaucoma screening.

In this paper, we propose then a new method for glaucoma screening and diagnosis (see \mbox{section \ref{proposed_method}}) from fundus images. Firstly, an OD detection approach is performed, allowing to detect the area of interest around the OD, even in the eventual presence of bright lesions in pathological cases (see \mbox{section \ref{detection}}). Here, a brightness criterion merged to a template matching technique is operated to effectively detect the disc, and extract the sub-window around it. Secondly, the OC and OD segmentation is operated using an unsupervised texture-based method, to effectively detect the pixels belonging to the desired areas without the computation of a complex learning phase. Then, a model-based boundary fitting method is subsequently applied, to improve segmentation performance (see \mbox{section \ref{segmentation}}). CDR calculation (see \mbox{section \ref{cdr_calculation}}) is finally used for glaucoma screening (see \mbox{section \ref{glaucoma_screening}}). The approach performance on final screening and diagnosis is evaluated and compared to the existing methods for glaucoma screening. Our proposed pipeline (see \mbox{Fig. \ref{diagramme}}) improves over the current state-of-the-art, while performing low-complex unsupervised algorithms.
